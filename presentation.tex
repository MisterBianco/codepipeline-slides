\documentclass{beamer}
\usepackage{hyperref}

\usetheme{metropolis}

\title{AWS Codepipelines}
\date{\today}
\author{Jarad Dingman}

\begin{document}
  \maketitle

  \section{About Me}

    \begin{frame}{About Me}
      I am a \textit{senior Devops Engineer} at \textbf{Western Governors University}, 
      I was a student at Dixie State from 2015-2018 (Francoms Favorite) and am pursuing 
      my Bachelors in CS and IT.

      \textbf{Socials}:
      \begin{itemize}
        \item @Misterbianco (Github)
        \item @Jarad.Dingman (Linkedin) - Send me a connection request
        \item I don't use other social medias and neither should you.
      \end{itemize}
    \end{frame}

    \begin{frame}{Note}
      These slides were created using \LaTeX and are available on my Github
      feel free to download and compile them at: 
        \url{https://github.com/misterbianco/codepipeline-slides}
    \end{frame}

  \section{Disclaimer}

    \begin{frame}{Disclaimer}
      While I believe AWS has many great offerings I am still rather divided
      on how much I can recommend Codepipelines due to the lack of certain key
      features including:

      \begin{itemize}
        \item Filtering on commit message
        \item Filtering on changed files
        \item Filtering on release and/or tagging
        \item Spawn time (very hit and miss)
      \end{itemize}

      Thus I have been actively helping my team use Github Actions where possible.
    \end{frame}

    \begin{frame}{Disclaimer}
      \textit{WGU is very early in its devops journey and is far from mature.
      I will explain how we do things and where necessary discuss a better way
      that we havent adopted yet.}
    \end{frame}

  % Begin actual presentation ----------------------------------

  \section{Codepipeline}

    \begin{frame}{How WGU uses Codepipeline}
      As is typical with many if not most organizations, we \textbf{DO NOT} 
      allow users, especially not software engineers to push code directly 
      into production. That would be catastrophic and have a severe impact
      on our SLA's.

      \vfill
      \textit{Instead we use codepipeline to deploy our infrastructure and code}
    \end{frame}

  \section{Service Catalog}
    \begin{frame}{Service Catalog}
      Service catalog is used to deploy pipelines.

      This service allows the creation of templates that developers can
      select as necessary to create their pipelines and infra.

      \vfill
      \textbf{Better way}: Infrastructure as Code
    \end{frame}

    \begin{frame}{Pipelines}
      After a developer has selected their template and provided the
      required information they must create their teams infrastructure.

      These components are their:
      \begin{itemize}
        \item application cluster
        \item messaging queue
        \item route53 (dns) records
        \item databases
      \end{itemize}
    \end{frame}

    \begin{frame}{Pipelines II}
      With your teams infra setup and the pipeline started the team can now
      focus on writing their applications.

      Pipelines at WGU typically are composed of the following stages:
      \begin{itemize}
        \item Source
        \item Build
        \item Scan
        \item Deploy
        \item Test
        \item Reroute (in the case of Blue-Green Deployments)
      \end{itemize}
    \end{frame}

  \section{Stages}
    \begin{frame}{Source}
      This stage is self explanatory, we clone the repo at the 
      given commit and pass it as an artifact to subsequent stages.
    \end{frame}

    \begin{frame}{Build}
      In this stage we accept the artifact from our previous stage
      and we use a bash script present in the codebase to build the
      application.
    \end{frame}

    \begin{frame}{Scan}
      In this stage we take the built artifact and we leverage tools
      to scan the artifact for vulnerabilities and other issues.
    \end{frame}

    \begin{frame}{Deploy}
      If we passed the scanning stage we then deploy the application
      into its respective environment.
    \end{frame}

    \begin{frame}{Test}
      Depending on your application type we do various rounds of testing.
      The most common application type is a web application and thus we
      run tests using postman.
    \end{frame}

  \section{Demo Time}

\end{document}